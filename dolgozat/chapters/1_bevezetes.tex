\Chapter{Bevezetés}

% TODO: Feature extraction-t, tesztelést, validálási módokat érdemes hangsúlyozni majd.

Annak ellenére, hogy széles körben elterjedt az elektronikus formában való megjelenítés, a kézzel írt anyag átalakítása elektronikus formában továbbra is nagy gazdasági jelentőségű problémát jelent.

Bár léteznek kielégítő megoldások a nyomtatott dokumentumok átalakítására, különösen a római ábécé számára. A kézírással kapcsolatos megoldások még mindig sok kihívással szembesülnek. 

A különböző nyelvek felismerésére szolgáló módszerek tekintetében sok hasonlóságot mutat a képfeldolgozás és a dokumentum előfeldolgozás, például zajcsökkentés, jellemző kinyerés, vonal szegmentálás stb. A felismerési módszerek feltétlenül eltérnek egymástól.

Az kézírásos kínai karakter felismerés alapvetően nehéz felismerési probléma. Alapvetően a karakterek száma (osztályok) nagyon magas, sok karakter nagyon összetett szerkezetű egyesek nagyon hasonlónak tűnnek, és a pontos szegmentálás nagyon sok helyzetben nagyon nehéz.

A dolgozatom során a kínai kézírási felismerés módszertanját szeretném bemutatni. Ennek megközelítéséhez több aktuális irodalmat dolgoztam fel. A dolgozatom bemutatja a kínai karakterek alapvető felépítését. Az optikai karakter felismerő (OCR) rendszerek működését.

Különösebb figyelmet fordítottam a kínai karakterek jellemzőinek kinyerésére. A jellemző kinyerés fontos feladat, mert ezek a technikák használatával hatékonyabb gépi tanulást érhetünk el.

Továbbá a jellemző kinyerés segít a fejlesztőknek, hogy csak a legfontosabb és hasznos adatokat használják a gépi tanulás során, ami drasztikusan csökkenti a költségeket és az adatmennyiséget.

Majd részletezem a konvolúciós hálózat, ami különös nagy áttörés az objektum felismerés témakörében. A rendszer nem csak a kínai karakterek felismerésére használható, hanem bármilyen más problémakör megoldható vele. A bemeneti képektől függ a hálózat tanulása. Az egyetlen hátránya a tanításhoz szükséges adatok mérete, de manapság egyre több adathalmaz válik publikussá.
\Chapter{Bevezetés}

Annak ellenére, hogy az esetek jelentős részében már elektronikus formában kezeljük a dokumentumokat, még mindig fontos szerepük van a papír alapú és szkennelt dokumentumoknak. Ezek elektronikussá alakítása továbbra is nagy gazdasági jelentőségű problémát jelent.

Bár léteznek kielégítő megoldások a nyomtatott dokumentumok átalakítására, különösen a kézírással írott szövegek digitalizálása még mindig komoly kihívást jelent.

A különböző nyelvű karakterek felismerése alapvetően képfeldolgozási és gépi tanulási módszerekre épülnek. Ezek lényeges lépései például a zajcsökkentés, a jellemzők kinyerése, vonalak szegmentálása. A felismerési módszerek nyelvenként és karakterkészletenként lényeges eltéréseket mutatnak.

Az kézírásos kínai karakter felismerés alapvetően nehéz felismerési probléma. A karakterek száma (így a későbbiekben az osztályok száma) nagyon nagy, sok karakter különösen összetett szerkezetű. Egyesek nagyon hasonlónak tűnnek, és a pontos szegmentálás többnyire nagyon nehéz.

A dolgozatom során a kínai kézírási felismerés módszertanát szeretném bemutatni. Ennek megközelítéséhez több aktuális irodalmat dolgoztam fel. A dolgozatom bemutatja a kínai karakterek alapvető felépítését, majd az optikai karakter felismerő (OCR -- \textit{Optical Character Recognition}) rendszerek működésére térek ki.

Különösebb figyelmet fordítottam a kínai karakterek jellemzőinek kinyerésére. A jellemző kinyerés fontos feladat, mert ezek adják azt a keresési teret, amelyben a gépi tanulásos módszerek az osztályozási problémát meg tudják oldani. Továbbá a jellemző kinyerés segít a fejlesztőknek, hogy csak a legfontosabb, a későbbi feldolgozáshoz hasznos adatokat használják a gépi tanulás során, ami drasztikusan csökkenti számítási és tárolási költségeket.

Részletezem a konvolúciós mesterséges neurális hálózatokat (CNN -- \textit{Convolutional Neural Network}), amely különös nagy áttörés az objektumfelismerés témakörében. A rendszer nem csak a kínai karakterek felismerésére használható, hanem bármilyen más problémakör megoldható vele. A bemeneti képektől függ a hálózat tanulása. Az egyetlen hátránya a tanításhoz szükséges adatok mérete, de manapság egyre több adathalmaz válik publikusan elérhetővé. A dolgozatomban elsősorban ezen hálózat típus alkalmazási módját mutatom be kínai karakterek felismerése kapcsán.

A kínai karakterek felismerésével már korábban egy TDK dolgozat formájában foglalkoztam \cite{tdk}. Abban elsősorban a tanításhoz használt minták generálási módjával foglalkoztam. Jelen dolgozat a jellemzők kinyerésére összpontosít.
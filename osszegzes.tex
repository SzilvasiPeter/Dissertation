\documentclass[a4paper,12pt]{article}

% Set margins
\usepackage[hmargin=3cm, vmargin=3cm]{geometry}

\frenchspacing

% Language packages
\usepackage[utf8]{inputenc}
\usepackage[T1]{fontenc}
\usepackage[magyar]{babel}

% AMS
\usepackage{amssymb,amsmath}

% Graphic packages
\usepackage{graphicx}

% Colors
\usepackage{color}
\usepackage[usenames,dvipsnames]{xcolor}

% Enumeration
\usepackage{enumitem}

% Links
\usepackage{hyperref}

\linespread{1.2}

\begin{document}

\pagestyle{empty}

\section*{Összegzés}

\textit{Szilvási Péter: Kínai karakterek felismerése konvolúciós neurális hálók használatával}

\bigskip

A dolgozatomban ismertetésre került a kínai karakterek felépítése, írási módjuk, bemutatva azok építőelemeit az alapvonásokat (\textit{strokes}). Ezt követően bemutattam a vonásrendek szabályait, amely hasznos az online karakterfelismerésnél.

Részletezésre került az optikai karakterfelismerés (OCR) müködése, annak részei. Mivel a probléma már rég óta közismert, ezért áttekintésre kerültek a manapság használt OCR-es megoldások, amelyek képesek felismerni a különböző betűtípusokkal nyomtatott kínai karaktereket.

A jellemző kinyerés fontos szakasz az osztályozás elött. A dimenziószám redukálása csökkenti a futási időt a későbbi szakaszokban. Ez lehetővé teszi a magasabb összetettségű algoritmusok használatát, több hiperparaméter keresését, vagy több értékelés elvégzését. A fontos jellemzők megtalálása kulcsfontosságú az osztályozás során.

A tématerület bemutatásához szükségesnek látszott a neurális hálózat alapvető elemeinek, tanítási és tesztelési módjának bemutatása. Ezt követően a képfelismeréshez leginkább ajánlott (az elérhető irodalmak alapján vélhetően a leghatékonyabb) konvolúciós neurális hálózatra (CNN) esett a választás. Az ezzel foglalkozó fejezet kifejti a CNN rétegeinek működését, továbbá bemutat egy aránylag aktuális kutatási eredményt, amely a felismerés pontosságát hasonlítja össze különböző hálózat architektúrák szerint.

A tervezés során bemutattam a \textit{Keras} könyvtárat. Eleinte bonyolultnak tűnő konvolúciós és hagyományos neurális hálózat könnyedén implementálható a \textit{Keras} segítségével. A megfelelő csomagok importálása után részleteztem a rétegek müködését. Egy egyszerű architektúrát építettem ki.

A validáció során betekintést nyerhetünk az eredmények ellenőrzéséhez összeállított adathalmazba, és hogy hogyan változik a hálózat osztályozási hatékonysága a bemeneti képek és a zajjal való terhelés hatására. Összességében tehát a generált, zajjal terhelt mintákkal történő tanítási módszer jól kombinálható a konvolúciós neurális háló alapú osztályozási módszerrel.

\end{document}

